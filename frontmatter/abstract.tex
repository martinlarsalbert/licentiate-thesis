
% -- Importance
Twin models are an important tool to make assessments of the real physical objects. The twin model comes in handy to assess things that are either too expensive, too dangerous or if the real object is unavailable, as in the case when NASA build a twin rocket for the Apollo missions. 
It is common today, that operational data is recorded onboard ships within the Internet of Ships paradigm which enables to build Ship Digital Twin (SDT) as a digital copy of the real ship. SDT is a fairly new concept with an increasing number of research publications where many new methods for system identification is proposed.
% -- Objective
\noindent This thesis proposes how the ship rigid body dynamics of the SDT can be identified. 
% -- Method, setting
\noindent The SDTs are developed with scale model test data from a controlled laboratory environment which is similar to the sea environment but with less noise and uncertainties. This is intended as an important gateway towards improving the SDTs for real sea conditions. 
% -- Results
\noindent It is shown that multicollinearity is a large problem, especially for the complex manoeuvring models, where the appropriate complexity needs to be selected with the bias-variance tradeoff between underfitting or overfitting the data.
% -- Conclusion 
\noindent The SDT models identified with the proposed methods generalizes well to unseen data for the test cases used in the thesis.

\vspace{0.1cm}
\textbf{Keywords:} Ship Digital Twin, Ship Manoeuvring, Parameter Identification Technique, Inverse Dynamics, Extended Kalman Filter, RTS smoother, Multicollinearity


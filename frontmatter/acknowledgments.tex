This thesis presents research performed since February 2020 at the Division of Marine Technology, Department of Mechanics and Marine Sciences at Chalmers University of Technology and SSPA Sweden AB (\href{www.sspa.se}{www.sspa.se}). Financial support for this research was provided by the DEMOPS project (Development of Methods for Operational Performance of Ships) funded by the Swedish Transport Administration (project: FP4 2020) and the D2E2F project (Data Driven Energy Efficiency of Ships) funded by the Swedish Energy Agency (project: 49301-1).

I had unsuccessfully been applying for funding of my PhD studies for a couple of years when Professor Wengang Mao contacted me three years ago with the offer to become a PhD student. Otherwise, I would probably be still searching for funding. I will always be very grateful for this opportunity. Wengang has also been my main supervisor during my studies, a guide in academic research, and a tutor in statistical and machine learning methods.  

This gratitude also goes to my examiner and co-supervisor, Professor Jonas W. Ringsberg,
head of the Division of Marine Technology. I have enjoyed our discussions about research methodology and how to organize a paper in academic writing, where his detailed proofreading has also been a great asset.

I also want to thank SSPA Sweden AB for allowing me to be an industrial PhD student within my current employment. A special thanks to Dr. Christian Finnsgård, head of the Research Department at SSPA, for his support and good advice throughout the project. I also want to mention all the personnel at SSPA who have been involved in model tests, building the ship models, and conducting the experiments.

\vskip 2pc

\noindent \thesisauthor

\noindent \thesiscity, November\  2022  % Since dedication is written a month or more before the actual thesis date, \thesismonth and \thesisyear is not used here.

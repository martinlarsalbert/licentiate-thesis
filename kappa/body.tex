%!TEX root = ../main.tex
%%%%%%%%%%%%%%%%%%%%%%%%%%%%%%%
%%%%%%%%%%%%%%%%%%%%%%%%%%%%%%%
\chapter{Introduction}
%%%%%%%%%%%%%%%%%%%%%%%%%%%%%%%
\section{Background}
Modeling of ship’s dynamics has a wide range of useful applications. This can be categorized as: white-box, black-box or grey-box modeling \cite{leifsson_grey-box_2008}. Most of the models in this thesis fall under the grey-box category.

\begin{itemize}
    \item White-box modeling \\
    involves applying physical principles, so that no observed data is required. Computational Fluid Dynamics (CFD) is one example from ship hydrodynamics. Semi-empirical models where unknown physical constants have been derived from previous experiments, can also be considered as white-box models \cite{leifsson_grey-box_2008}.  

    \item Black-box modeling \\
    means that parameters do not have physical significance but where the objectives is to find a good model that fits the observed data \cite{lindskog_tools_1995}.
    
    \item Grey-box modeling \\
    is a combination of white-box and black-box modeling methods, so that observed data is required. This concept is also referred to as semi-physical modeling, hybrid modeling or semi-mechanistic modeling \cite{leifsson_grey-box_2008}. 
\end{itemize}

\noindent The black-box modeling is entirely data driven, which means that no prior understanding of the system generating the data is needed. The main disadvantage is the dependence on the data used to model the system, which can result in limited extrapolation properties beyond the data that it is derived from \cite{leifsson_grey-box_2008}. 
In a grey box model the white and black parts can be combined in several ways using either a serial or parallel approach \cite{leifsson_grey-box_2008}. 
Grey-box modelling is often used in situations where white-box models are not giving the required accuracy by introducing some corrections before or after the white-box in a serial approach. 
Grey-box modeling of a motorcycle shock absorber \cite{beghi_grey-box_2007} is an example of the parallel approach where the low frequency dynamics is handled by a white box and the higher frequencies are handled by a black box.

\section{Literature review}


%"Critic" to what has been done before
\section{Motivation and objective}

crawling, walking, running...

\section{Assumptions and limitations}

\section{Outline of the paper}


%%%%%%%%%%%%%%%%%%%%%%%%%%%%%%%
%%%%%%%%%%%%%%%%%%%%%%%%%%%%%%%
\chapter{Methods\label{ch:methods}}
%%%%%%%%%%%%%%%%%%%%%%%%%%%%%%%
The ship's dynamics comprises the forces and motions in the six degrees of freedoms (6DOF): surge, sway, heave, roll, pitch and yaw. Heave and pitch motions are often neglected in calm water conditions, so that only four degrees of freedom (4DOF) model is sufficient to express the ship's dynamics. Physically parameterized models for roll and the Vessel Manoeuvring Model (VMM) for the remaining DOFs are presented in section \ref{sec:roll} and section \ref{sec:VMM}. How the parameters in these models can be identified is presented in \ref{sec:PIT_roll}.

\section{Roll motion} \label{sec:roll}

The roll motion on a straight course in calm water with no external forces can be expressed with Eq.\ref{eq:roll_decay_equation_general_himeno} \cite{himeno_prediction_1981},
\begin{equation} \label{eq:roll_decay_equation_general_himeno}
A_{44} \ddot{\phi} + \operatorname{B_{44}}\left(\dot{\phi}\right) + \operatorname{C_{44}}\left(\phi\right) = 0
\end{equation}


\noindent where $B_{44}\left(\dot{\phi}\right)$ can be expressed as expansion series:  
$ B_{44}\left(\dot{\phi}\right) = B_1\cdot\dot{\phi} + B_2\cdot\dot{\phi}\left|\dot{\phi}\right| + B_3\cdot\dot{\phi}^3 + ... + B_n\cdot\dot{\phi}^n$. Most often, the so-called ``linear model'' (Eq.\ref{eq:roll_decay_equation_himeno_linear}), ``quadratic model'' (Eq.\ref{eq:roll_decay_equation_himeno_quadratic_b}) and ``cubic model'' (Eq.\ref{eq:roll_decay_equation_cubic}) are used to represent $B_{44}(\dot{\phi})$ in by truncating the series to keep only linear, quadratic and cubic terms,

\begin{equation} \label{eq:roll_decay_equation_himeno_linear}
A_{44} \ddot{\phi} + B_{1} \dot{\phi} + C_{1} \phi = 0
\end{equation}

\begin{equation} \label{eq:roll_decay_equation_himeno_quadratic_b}
A_{44} \ddot{\phi} + C_{1} \phi + \left(B_{1} + B_{2} \left|{\dot{\phi}}\right|\right) \dot{\phi} = 0
\end{equation}

\begin{equation} \label{eq:roll_decay_equation_cubic}
A_{44} \ddot{\phi} + \left(B_{1} + B_{2} \left|{\dot{\phi}}\right| + B_{3} \dot{\phi}^{2}\right) \dot{\phi} + \left(C_{1} + C_{3} \phi^{2} + C_{5} \phi^{4}\right) \phi = 0
\end{equation}


\section{Vessel Manoeuvring Model} \label{sec:VMM}

\section{Parameter Identification Techniques (PIT)}

\subsection{Roll damping Parameter Identification} \label{sec:PIT_roll}
\noindent A Parameter Identification Technique (PIT) can be applied to identify the roll damping parameters ($B_1$, $B_2$, $B_3$) and stiffness parameters ($C_1$, $C_3$, $C_5$) in the parameterized roll motion models in Eq.\ref{eq:roll_decay_equation_himeno_linear}, Eq.\ref{eq:roll_decay_equation_himeno_quadratic_b} and Eq.\ref{eq:roll_decay_equation_cubic}. These equations do not have unique solutions, considering that the whole equations can be multiplied by an arbitrary factor to obtain new valid solutions. The inertia is therefore excluded, to obtain unique solutions, by normalizing the equations by the total roll inertia $A_{44}$.
The normalized damping and stiffness parameters identified by a PIT can then be expressed in dimensional units by multiplication of $A_{44}$. If $A_{44}$ is not known before hand, it can be calculated using Eq.\ref{eq:A_44_eq} \cite{piehl_ship_2016}, assuming that the $GM$ is known.
\begin{equation} \label{eq:A_44_eq}
A_{44} = \frac{GM g m}{\omega_{0}^{2}}
\end{equation}

\noindent The frequency $\omega_0$ can be obtained with Fast Fourier Transform (FFT) of the roll signal. 

Two different PIT methods have been investigated: the ``derivation approach'' (referred to as PIT in \parencite{imo_1200_2006}) and the ``integration approach'' which is similar to what \parencite{soder_assessment_2019} used. In the derivation approach the first and second roll time derivatives are calculated numerically so that the parameters in the models are the only unknowns. A least squares fit could be used on the roll motion equation to identify any parameter, including nonlinear or frequency parameter. In the integration approach, the parameters are found by solving a nonlinear problem using the least-square method. This approach requires that an ordinary differential equation to be solved for many estimated sets of parameters until the solution converges.

%%%%%%%%%%%%%%%%%%%%%%%%%%%%%%%
%%%%%%%%%%%%%%%%%%%%%%%%%%%%%%%
\chapter{Results\label{ch:results}}
%%%%%%%%%%%%%%%%%%%%%%%%%%%%%%%

\section{Paper \ref{pap:rolldamping}}
\subsection*{"\nameref{pap:rolldamping}"}
A serial grey-box model for ship roll damping is developed in Paper \ref{pap:rolldamping}. Simplified Ikeda's (SI) method \cite{kawahara_simple_2011} is used as the white box model, which is combined with a following black-box correction model.
A roll damping dataset is used to train the black-box part of the grey-box model.
250 roll decay tests assembled from the Maritime Dynamics Laboratory at SSPA Sweden AB (\href{www.sspa.se}{www.sspa.se}) are used to build the dataset. The roll damping parameters are found by applying a PIT (see Section \ref{sec:PIT_roll}) on the parameterized roll motion models (see Section \ref{sec:roll}).
Both the ''derivation approach'' and ''integration approach'' was attempted for the roll damping PIT, where the ''integration approach'' gave the most accurate models, and was therefore used to build the roll damping dataset.

\section{Paper \ref{pap:daiyong}}
\subsection*{"\nameref{pap:daiyong}"}

\section{Paper \ref{pap:pit}}
\subsection*{"\nameref{pap:pit}"}

%%%%%%%%%%%%%%%%%%%%%%%%%%%%%%%
%%%%%%%%%%%%%%%%%%%%%%%%%%%%%%%
\chapter{Conclusions\label{ch:conclusions}}
%%%%%%%%%%%%%%%%%%%%%%%%%%%%%%%
The PIT ''derivation approach'' that was attempted in Paper \ref{pap:rolldamping}, was again attempted for the VMM PIT in Paper \ref{pap:pit}, where the velocities and accelerations are estimated with the EKF + RTS smoother, instead of the numeric differentiation that was used in Paper \ref{pap:rolldamping}.

%%%%%%%%%%%%%%%%%%%%%%%%%%%%%%%
%%%%%%%%%%%%%%%%%%%%%%%%%%%%%%%
\chapter{Future work\label{ch:future_work}}
%%%%%%%%%%%%%%%%%%%%%%%%%%%%%%%

%%%%%%%%%%%%%%%%%%%%%%%%%%%%%%%
%%%%%%%%%%%%%%%%%%%%%%%%%%%%%%%
\chapter{Conclusions\label{ch:conclusions}}
%%%%%%%%%%%%%%%%%%%%%%%%%%%%%%%
The PIT ''integration approach'' produced better models than the ''derivation approach'' for the roll motion models in Paper \ref{pap:rolldamping}. The ''integration approach'' is very slow and relies on optimization that may or may not converge.
The numerical differentiation that was used in Paper \ref{pap:rolldamping} to estimate the velocities and accelerations, is believed to be the main cause of the poor performance of the much faster and more reliable ''derivation approach''. A similar issue was also encountered in Paper \ref{pap:daiyong}.
The introduction of EKF + RTS smoother in the PIT presented in Paper \ref{pap:pit}, seems to be a good way to solve this issue.

The SI method, being the white-box physical model in the grey-box model in Paper \ref{pap:rolldamping} has about the same accuracy as the corresponding black-box model, which means that the white-box model is adding very little value, due to poor performance of SI method outside its limits.

 

   
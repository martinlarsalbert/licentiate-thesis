%%%%%%%%%%%%%%%%%%%%%%%%%%%%%%%
%%%%%%%%%%%%%%%%%%%%%%%%%%%%%%%
\chapter{Conclusions\label{ch:conclusions}}
%%%%%%%%%%%%%%%%%%%%%%%%%%%%%%%
The main findings and conclusions are presented below with respect to the research activities described in Section \ref{sec:motivation}.

\subsubsection*{Develop a method to handle noise}
The PIT ''integration approach'' produced better models than the ''derivation approach'' for the roll motion models in Paper \ref{pap:rolldamping}. The ''integration approach'' is very slow and relies on optimization that may or may not converge.
The numerical differentiation that was used in Paper \ref{pap:rolldamping} to estimate the velocities and accelerations, is believed to be the main cause of the poor performance of the much faster and more reliable ''derivation approach''. A similar issue was also encountered in Paper \ref{pap:daiyong}, where the identified hydrodynamic derivatives were very sensitive to the choice of the regularisation factor of the LS-SVR.
The iterative EKF + RTS smoother proposed in Paper \ref{pap:pit}, seems to solve these issues.

\subsubsection*{Propose a methodology to choose the appropriate and robust VMM for extrapolation}
A methodology to select a suitable VMM based on cross-validation is proposed in Paper \ref{pap:pit}. In order to investigate the model's ability to extrapolate and make predictions outside its training data, the validation set should have larger yaw rates, drift angles and rudder angles compared to the training set. The methodology was applied on the wPCC and KVLCC2 test cases, where turning circles where predicted with good accuracy on models trained on zigzag tests. 

\subsubsection*{Develop a PIT for three degrees of freedom (surge, sway and yaw)}
The challenges in identifying the hydrodynamic derivatives of the AVMM on noisy experimental data was shown in Paper \ref{pap:daiyong}. The challenges were mitigated in Paper \ref{pap:pit}, by improved preprocessing of the noisy data and by reduced multicollinearity of the VMM. The multicollinearity of the AVMM was reduced, by excluding hydrodynamic derivatives and the introduction of a propeller thrust model.

\subsubsection*{Develop grey-box model for prediction of modern ship roll             damping parameters}
The SI method, being the white-box physical model in the grey-box model in Paper \ref{pap:rolldamping} has about the same accuracy as the corresponding black-box model, which means that the white-box model is adding very little value, due to poor performance of SI method outside its limits.

\subsubsection*{Develop a PIT for roll motion for roll damping database}
The PIT identified the parameters in the quadratic roll motion model (Eq.\ref{eq:roll_decay_equation_himeno_quadratic_b}) with good accuracy for the 250 roll decay tests obtained from SSPA. Where the linear model (Eq.\ref{eq:roll_decay_equation_himeno_linear}) was ruled out as too simple and the cubic model (Eq.\ref{eq:roll_decay_equation_cubic}) as unnecessary complex.   

%%%%%%%%%%%%%%%%%%%%%%%%%%%%%%%
%%%%%%%%%%%%%%%%%%%%%%%%%%%%%%%
\chapter{Future work\label{ch:future_work}}
%%%%%%%%%%%%%%%%%%%%%%%%%%%%%%%

\subsubsection*{New generic roll damping model}
The PIT of the roll motion models applied on 250 roll-decay tests that was investigated in Paper \ref{pap:rolldamping} produced a roll damping database for modern ship types. A grey-box model and a black-box model as described in Section \ref{sec:genericrolldampingmodel} was developed from this database. The prediction accuracy of these models were not very good and should be improved. This is important, especially since the SI method, being the state of art white-box physical model for roll damping, was found to be outside its limits for the ships in the database. This implies that there is a need for a new roll damping prediction method for modern ships.

\subsubsection*{Model for Sailing Digital Twin}

\subsubsection*{Black box model for manoeuvring}
It has been shown in this thesis that grey-box models for the ship rigid body dynamics in calm waters can be developed with high accuracy and generalization. Can Black-box models give better or worse results, with respect to accuracy an generalization?  
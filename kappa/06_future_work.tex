%%%%%%%%%%%%%%%%%%%%%%%%%%%%%%%
%%%%%%%%%%%%%%%%%%%%%%%%%%%%%%%
\chapter{Future work\label{ch:future_work}}
%%%%%%%%%%%%%%%%%%%%%%%%%%%%%%%

\subsubsection*{New generic roll damping model}
The  of the roll motion models applied on 250 roll-decay tests that was investigated in Paper \ref{pap:rolldamping} produced a roll damping database for modern ship types. A grey-box model and a black-box model as described in Section \ref{sec:genericrolldampingmodel} was developed from this database. The prediction accuracy of these models were not very good and should be improved. This is important, especially since the SI method, being the state of art white-box physical model for roll damping, was found to be outside its limits for the ships in the database. This implies that there is a need for a new roll damping prediction method for modern ships.  

\subsubsection*{Black-box model}
Investigate black-box models, as an alternative to the grey-box models proposed in this thesis and compare with respect to accuracy and generalization.

\subsubsection*{Model for sailing digital twin}
Develop models from wPCC sailing model tests for the future sailing ship digital twins.

\subsubsection*{Manoeuvring system identification on 100 ships}
Expanding the analysis of the two ships from Paper \ref{pap:pit} to 100 ships tested at SSPA Sweden AB.

\subsubsection*{System identification of ship rigid body dynamics in wind and waves}

\subsubsection*{System identification of ship rigid body dynamics in full scale calm waters}

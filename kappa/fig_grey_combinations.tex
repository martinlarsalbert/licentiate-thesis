\begin{figure}[H]
    \centering
    \begin{subfigure}[b]{0.3\textwidth}
    \centering
    \begin{tikzpicture}[node distance=2cm]
    \node (white-box) [white-box] {\footnotesize White-box};
    \node (black-box) [black-box, right of=white-box, xshift=2cm] {\footnotesize Black-box};
    \draw [arrow] (white-box) -- (black-box);
    \end{tikzpicture}
    \caption{Serial grey-box.}
    \label{fig:serial1}
    \end{subfigure}

    
    \begin{subfigure}[b]{0.3\textwidth}
    \centering
    \begin{tikzpicture}[node distance=2cm]
    \node (black-box) [black-box] {\footnotesize Black-box};
    \node (white-box) [white-box, right of=black-box, xshift=2cm] {\footnotesize White-box};
    \draw [arrow] (black-box) -- (white-box);
    \end{tikzpicture}
    \caption{Serial grey-box.}
    \label{fig:serial2}
    \end{subfigure}

    \begin{subfigure}[b]{0.3\textwidth}
    \centering
    \begin{tikzpicture}[node distance=2cm]
    \node (black-box) [black-box] {\footnotesize Black-box};
    \node (white-box) [white-box, below of=black-box] {\footnotesize White-box};
    \node (join) [process, right of=black-box, xshift=2cm, yshift=-1cm] {\footnotesize Join};
    \draw [arrow] (black-box) -- (join);
    \draw [arrow] (white-box) -- (join);
    \end{tikzpicture}
    \caption{Parallel grey-box.}
    \label{fig:parallel}
    \end{subfigure}
    \caption{Several ways to combine white- and black-box models in grey box models.}
    \label{fig:greycombinations}
\end{figure}